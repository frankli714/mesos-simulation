% THIS IS SIGPROC-SP.TEX - VERSION 3.1
% WORKS WITH V3.2SP OF ACM_PROC_ARTICLE-SP.CLS
% APRIL 2009
%
% It is an example file showing how to use the 'acm_proc_article-sp.cls' V3.2SP
% LaTeX2e document class file for Conference Proceedings submissions.
% ----------------------------------------------------------------------------------------------------------------
% This .tex file (and associated .cls V3.2SP) *DOES NOT* produce:
%       1) The Permission Statement
%       2) The Conference (location) Info information
%       3) The Copyright Line with ACM data
%       4) Page numbering
% ---------------------------------------------------------------------------------------------------------------
% It is an example which *does* use the .bib file (from which the .bbl file
% is produced).
% REMEMBER HOWEVER: After having produced the .bbl file,
% and prior to final submission,
% you need to 'insert'  your .bbl file into your source .tex file so as to provide
% ONE 'self-contained' source file.
%
% Questions regarding SIGS should be sent to
% Adrienne Griscti ---> griscti@acm.org
%
% Questions/suggestions regarding the guidelines, .tex and .cls files, etc. to
% Gerald Murray ---> murray@hq.acm.org
%
% For tracking purposes - this is V3.1SP - APRIL 2009

\documentclass{acm_proc_article-sp}

\usepackage{mathtools}
\usepackage{graphicx}

\begin{document}

\title{Market-based Mechanisms for Resource Scheduling}
%\subtitle{[Extended Abstract]
%\titlenote{A full version of this paper is available as
%\textit{Author's Guide to Preparing ACM SIG Proceedings Using
%\LaTeX$2_\epsilon$\ and BibTeX} at
%\texttt{www.acm.org/eaddress.htm}}}
%
% You need the command \numberofauthors to handle the 'placement
% and alignment' of the authors beneath the title.
%
% For aesthetic reasons, we recommend 'three authors at a time'
% i.e. three 'name/affiliation blocks' be placed beneath the title.
%
% NOTE: You are NOT restricted in how many 'rows' of
% "name/affiliations" may appear. We just ask that you restrict
% the number of 'columns' to three.
%
% Because of the available 'opening page real-estate'
% we ask you to refrain from putting more than six authors
% (two rows with three columns) beneath the article title.
% More than six makes the first-page appear very cluttered indeed.
%
% Use the \alignauthor commands to handle the names
% and affiliations for an 'aesthetic maximum' of six authors.
% Add names, affiliations, addresses for
% the seventh etc. author(s) as the argument for the
% \additionalauthors command.
% These 'additional authors' will be output/set for you
% without further effort on your part as the last section in
% the body of your article BEFORE References or any Appendices.

\numberofauthors{1} %  in this sample file, there are a *total*
% of EIGHT authors. SIX appear on the 'first-page' (for formatting
% reasons) and the remaining two appear in the \additionalauthors section.
%
\author{
% You can go ahead and credit any number of authors here,
% e.g. one 'row of three' or two rows (consisting of one row of three
% and a second row of one, two or three).
%
% The command \alignauthor (no curly braces needed) should
% precede each author name, affiliation/snail-mail address and
% e-mail address. Additionally, tag each line of
% affiliation/address with \affaddr, and tag the
% e-mail address with \email.
%
% 1st. author
\alignauthor Paul Christiano, Frank Li, Richard Shin \\
% 2nd. author
% 3rd. author
\email{\{paulfchristiano, frankli, ricshin\}@cs.berkeley.edu}
%\and  % use '\and' if you need 'another row' of author names
%% 4th. author
%\alignauthor Lawrence P. Leipuner\\
%       \affaddr{Brookhaven Laboratories}\\
%       \affaddr{Brookhaven National Lab}\\
%       \affaddr{P.O. Box 5000}\\
%       \email{lleipuner@researchlabs.org}
%% 5th. author
%\alignauthor Sean Fogarty\\
%       \affaddr{NASA Ames Research Center}\\
%       \affaddr{Moffett Field}\\
%       \affaddr{California 94035}\\
%       \email{fogartys@amesres.org}
%% 6th. author
%\alignauthor Charles Palmer\\
%       \affaddr{Palmer Research Laboratories}\\
%       \affaddr{8600 Datapoint Drive}\\
%       \affaddr{San Antonio, Texas 78229}\\
%       \email{cpalmer@prl.com}
}
% There's nothing stopping you putting the seventh, eighth, etc.
% author on the opening page (as the 'third row') but we ask,
% for aesthetic reasons that you place these 'additional authors'
% in the \additional authors block, viz.
%\additionalauthors{Additional authors: John Smith (The Th{\o}rv{\"a}ld Group,
%email: {\texttt{jsmith@affiliation.org}}) and Julius P.~Kumquat
%(The Kumquat Consortium, email: {\texttt{jpkumquat@consortium.net}}).}
%\date{30 July 1999}
% Just remember to make sure that the TOTAL number of authors
% is the number that will appear on the first page PLUS the
% number that will appear in the \additionalauthors section.

\maketitle
\begin{abstract}

\end{abstract}

% A category with the (minimum) three required fields
%\category{H.4}{Information Systems Applications}{Miscellaneous}
%A category including the fourth, optional field follows...
%\category{D.2.8}{Software Engineering}{Metrics}[complexity measures, performance measures]

%\terms{Theory}

%\keywords{ACM proceedings, \LaTeX, text tagging} % NOT required for Proceedings

\section{Introduction}
\label{sec:intro}

Large-scale computing clusters offer potential for greater efficiency through
pooling of computational resources, but must contend with the problem of
allocating these resources to users with competitive and potentially
incompatible demands. Such conflicts are typically arbitrated \emph{fairness}
criterions, such as dominant resource fairness (DRF), which benefit from their
simplicity, efficiency, and non-manipulability. Unfortunately, when applied in
realistic settings---when dealing with inhomogeneous resources, time-varying
demands, requirements for data locality, etc.---these systems cease to be either
efficient or non-manipulable.

We propose using prices as an abstraction for subsystems to communicate their
constraints, and describe a protocol based on auctions for allocating resources
in a cluster. Our protocol has the key advantage that it can continue to meet
efficiency guarantees in realistic environments where preferences are
inhomogeneous across resources and time, and in the presence of adversarial
manipulation. Moreover, we propose prices as a general mechanism for enforcing a
clean abstraction between different components of a system, allowing one
component to make choices intelligently based on constraints elsewhere in the
system even when it was not designed with knowledge of those constraints.

We implement these ideas in the Mesos cluster manager, along with microeconomic
``agents'' which allow traditional Mesos clients to interact seamlessly with our
modified implementation. We show that the computational and communication
overhead proposed by our system is manageable and that its performance compares
favorably to existing approaches (particularly DRF) even when users do not
provide any meaningful price information. Moreover, we show that when accurate
economic data is available our system is able to achieve substantially higher
social welfare than DRF. Finally, we discuss a number of examples where this
functionality may be useful even in applications where the end-users of the
system don't have to think about economics at all, and instead prices are
generated algorithmically as a principled way to accommodate constraints that
are currently managed in an ad hoc way.

\subsection{Guide to the paper}

In section \ref{sec:overview}, we provide a brief overview of Mesos and how it
could incorporate microeconomic mechanisms.

In section \ref{sec:case}, we make the theoretical case for using microeconomic methods. We
provide a litany of examples in which existing approaches, and in particular
DRF, fail to be either efficient or strategy-proof, and describe examples of
strategic behavior in Mesos. We discuss a number of examples in which economic
data generated \emph{within} the system could be useful, and compare this approach to a
number of ad hoc approaches currently pursued to solve similar problems.

In section \ref{sec:design}, we describe our protocol and the most important design decisions
we made. In particular, we describe the mechanisms we used to ensure that our
protocol remains backwards-compatible, and can be applied without any change to
the clients’ code.

In section \ref{sec:eval}, we report on simulated comparisons between our
design and existing approaches: (a) we evaluate the computational and communication
overhead of our approach and show that it is manageable, (b) we compare our
system in ``backwards-compatibility mode,'' without any economic information
whatsoever, to DRF, and show that it performs comparably whether judged by
utilization or fairness, (c) we examine a number of cases in which economic
information is available; we show that in general our algorithms achieve higher
social efficiency than DRF, and provide natural examples in which DRF fails to
be Pareto efficient.

In section \ref{sec:future}, we elaborate on directions for further
investigation.

In section \ref{sec:conclusion}, we making closing remarks.



\section{Overview}
\label{sec:overview}

We consider an environment in which a single cluster is used for a variety of
tasks, which may have very different demands for resources. For example, a
company might run research tasks in the same cluster as production web services,
in order to make use of computational resources which would otherwise lie idle
when web traffic is low. In general, running different tasks in the same cluster
has the promise of improving utilization and reducing overhead. 

Different tasks may have significantly different demands for resources; some may
have deadlines on the scale of seconds while others have deadlines of hours or
days, some may require specialized hardware available on only certain machines,
some may require using the same machines consistently while others can use
different machines at different times, some may have complex constraints on data
locality, etc. 

\subsection{Mesos}
The Mesos cluster manager is designed to address these problems, and to run
several independent \emph{frameworks} on a single cluster. Its core design philosophy
is to make minimal assumptions about the behavior of the frameworks it hosts. To
this end it allocates resources using ``resource offers,'' and leaves it up to the
frameworks themselves to decide which resources they will use and which they
will pass on. Frameworks are free to make these decisions on the basis of the
detailed specifications of the tasks they want done (for example, to accept
machines that improve data locality), but Mesos itself makes use of a very
simple fairness criterion to decide which frameworks receive offers.

The abstraction of resource offers allows Mesos to serve frameworks without
understanding the constraints they face. By making offers to those frameworks
which are currently using the least resources, Mesos attempts to ensure that all
frameworks receive a fair share of the cluster. Because of the simplicity of
this protocol, it appears to be immune to manipulation (and indeed it has been
shown to be immune to manipulation in theory).

Unfortunately, while the fairness policy used by Mesos is efficient and
strategy-proof in a simple one-shot game with homogeneous resources, it is
neither efficient nor strategy-proof in this setting. For example, a framework
using Mesos which prefers to run tasks on machines that already have relevant
data is forced to engage in strategic behavior: given a resource which it
\emph{could} use now, should it take the resource or decline it, so that it
might receive a better offer in the future? Solutions to this problem rely on
simple heuristics which require tuning based on the parameters of the system;
there are simple examples where the resulting allocations are not efficient
despite best-efforts of the clients, and in complex examples there is little to
believe the results are efficient. Moreover, despite being ``strategy-proof'' the
current design of Mesos is essentially unsuitable for even mildly hostile
environments; a framework which can use less than its ``fair share'' of the
resources at any point in time has no incentive to do so, and indeed has every
incentive to exploit all resources it can be allocated even if they are being
used arbitrarily inefficiently (and there is always \emph{something} to do with
idle resources, if only perform proof-of-work in exchange for small amounts of
money).

\subsection{Microeconomic methods}
In this paper we confront the challenge of modifying Mesos to provide more
robust efficiency guarantees, without compromising its agnosticism regarding the
nature of different frameworks' desiderata. We propose \emph{price signals} as a
mechanism for accomplishing this; instead of accepting or rejecting a resource
offer, a framework can express its preferences by assigning a \emph{willingness
to pay} for that resource, and Mesos can give the resources to whichever
frameworks are most willing to pay.

We stress the distinction between this use of price signals and the more typical
motivation. We are most interested in price signals as an \emph{abstraction} which
programs can use to express their constraints, rather than as a mechanism to
extract revenue for the cluster operator or to arbitrate between competing
external interests. To consider a simple case, suppose there are two programs
that are contending for a shared resource $R$. Each could either make use of $R$ or
some alternative resource $R'$ (for example, a different machine in the cluster at
a different time). Our proposal is to have each program articulate the value of
resource $R$ (compared to the benchmark of not receiving resource $R$, and then
contending for resource $R'$ at a later date). The primary alternatives that have
been considered are:
\begin{enumerate}
  \item Ignore the relative value of resource $R$ and $R'$, and make allocation
    decisions entirely on the basis of fairness considerations (for example by
    ensuring that each framework has an equal number of every resource). This is
    the approach taken by Mesos, although it in fact achieves better behavior
    precisely by \emph{violating} the strategy-proofness of its underlying allocation
    policy, and providing incentives to frameworks to strategically decline a
    resource in order to receive a better allocation in the future.
  \item Understand the entire optimization problem faced by each problem, and
    make explicit judgments of the relative value of $R$ and $R'$ within the
    scheduler. This violates the spirit of Mesos, and in general appears to be
    difficult given the complexity and scale of the cluster scheduling problem.
\end{enumerate}


\section{The case for microeconomic scheduling}
\label{sec:case}

\subsection{Scheduling is hard}
Consider the following typical situations within a cluster:
\begin{enumerate}
  \item Some clients have regular throughput and flexible deadlines, while some
    clients have hard real-time constraints and variable traffic.
  \item Different tasks make use of different characteristics of hardware on
    machines in the cluster, for example high ratios of RAM to CPU on a
    particular machine, or GPUs or other hardware that is available on only
    certain machines.
  \item Tasks require data which is located on disk or even in memory somewhere
    in the cluster, and co-locating the tasks with the data improves
    performance.
\end{enumerate}
In each of these examples, clients of the cluster have preferences over
allocations which are difficult to precisely express. It would be possible but
difficult to cover just these cases, but the reader can doubtless generate many
similar cases, and as the environment changes the nature of these constraints is
likely to continue to change as well. 

Mesos is able to schedule even in the face of such constraints, by making
resource offers to clients and allowing them to accept or reject resources
according to whether they satisfy their constraints. But this is approach offers
little granularity to clients, forces them to adopt brittle heuristics, and
moreover compromises the guarantees of the fairness policies Mesos uses to
decide how to make resource offers.

Mesos uses a fair-share scheduler based on dominant resource fairness. This
means that the client to receive the next resource offer is the one who is
currently using the minimum share of the resource they are using the \emph{most}
of (relative to the total supply in the cluster). A weighted version can be
defined analogously. 

Consider a client of Mesos who is interested in receiving a machine with
characteristic X, because it would allow their task to run 20\% faster (or
perhaps because it would allow them to run a task which is slightly higher
priority than the task they would run on a machine without characteristic X).
Given an offer for a machine without characteristic X, what should this client
do? The formal guarantee of ``strategy-proofness'' suggests the client would take
it; indeed, to do otherwise would explicitly be strategic. But in fact, we hope
that clients will yield a highly-contended resource when they can use an
alternative, or will yield a typical resource if they could make particularly
good use of an alternative. And indeed this is done by many clients to Mesos.

This example clearly shows that in this case, the mechanism provided by Mesos is
\emph{not} strategy-proof. Worse, it’s not Pareto efficient if people are
honest, and need not be Pareto efficient if clients behave optimally. Similar
things happen in the other examples we mentioned, or in most of the similar
alternatives which we can envision:
\begin{enumerate}
  \item When one client has a constant stream of work and another has varying
    traffic, Mesos encourages both to constantly use up their allocated share of
    the network (unless they have nothing to do with the space). But an
    efficient allocation would give the client with varying workload more
    resources when it has more work. 
  \item Mesos gives a valuable resource to whatever client happens to be using
    the fewest resources when that machine becomes available. This encourages
    clients to inefficiently decline resources which they could use, so that
    they will be offered better resources in the future. This may decrease
    utilization, and moreover it provides little guarantee that a valuable
    resource will be used by the framework which most values it---instead it will
    be used by whichever framework uses the least resources.
  \item If a client is waiting for a machine that is co-located with some data
    that their task needs, then Mesos encourages them to guess how likely that
    resource is to be freed up soon, and to decline new resources if this
    probability is high enough. This is actually much more efficient behavior
    than if the client behaved honestly by accepting whatever resources it could
    use, but it provides some indication that the ``guarantees'' of DRF are not
    directly applicable to the situations Mesos is intended to cover.
  \end{enumerate}

These shortcomings are not specific to DRF; these will be issues for any
scheduler which treats the scheduling problem as a one-shot game with
homogeneous resources when in fact it is more complicated. Theoretical
guarantees are not likely to be meaningful in this regime. To evaluate proposed
solutions, we suggest a combination of:
\begin{enumerate}
  \item Theoretical analyses which apply in realistic contexts
  \item Experimental evidence for effectiveness
\end{enumerate}
We provide strong evidence that our proposal is at least not inferior to DRF on
either of these grounds, and we provide some evidence that even our highly
preliminary implementation is already superior.

\subsection{Existing approaches}
Many difficulties with scheduling have been isolated and addressed, typically by
ad hoc mechanisms which require tuning and don't necessarily play nicely with
each other. In order to contrast these approaches with microeconomic methods, we
list some examples of solutions to these problems here. Note that we will not be
able to solve all of these problems within our proposal---that would be an
extraordinarily ambitious project. We list these examples to give some sense of
how a microeconomic framework could be used to replace ad hoc approaches,
improving robustness and ease-of-design. In contrast, our goal in this paper is
simply to show that microeconomic approaches really are a feasible option, and
that in the ``base case'' they are as good as existing solutions.

\paragraph{Deadlines} DRF makes no provision for coping with deadlines, which
are often of considerable practical importance. The Jockey system enforces
deadlines in the context of a fair-share scheduler, by increasing or decreasing
a task's share of the cluster depending on the importance of their deadline.
While this is a significant improvement over the alternative, we find it a very
crude solution to the underlying problem:
\begin{enumerate}
  \item Jockey has no dependence on how important resources currently are for
    the cluster, it just assigns a job as many resources as are needed to meet
    its deadline.
  \item It is not clear how to generalize Jockey to systems where multiple
    simultaneous tasks have deadlines, or where meeting deadlines must be
    balanced against any other consideration.
  \item Jockey changes the allocations of tasks depending on their performance,
    and will typically invalidate any guarantees provided by the original
    scheduler.
  \item Jockey does not have any principled approach for reasoning about the
    error bars on a task's completion time (instead this is determined by a
    fine-tuned parameter of the system).
  \item Jockey cannot reason about multiple deadlines of varying importance, but
    instead gives a job as many resources as it needs to finish as soon as it
    would be useful.
\end{enumerate}

The microeconomic alternative makes use of Jockey's core component, an estimate
for the time required to complete a task, but uses a different control system.
In the microeconomic framework a deadline is associated with an economic value
(that might be algorithmically generated based on the priority of the task)
which can be compared to the economic value of other activity in the cluster.
The importance of resources to the task is then given by the effect those
resources will have on the task's probability of completing by deadline, and
this can be compared to the importance of those resources to other processes
running on the cluster. Resources can then be shifted until these two values are
equal.

\paragraph{Priority and preemption} In existing systems priority is typically
implemented as an absolute ordering; a task of high priority can preempt a task
of lower priority and always wins in contention for resources. This approach is
conceptually problematic because there are always mechanisms for a
higher-priority process to improve performance by increasing its resource usage
(for example speculative execution, or increased duplication of straggling
processes) and an infinite tradeoff between the performance of high-priority and
low-priority processes is undesirable. In practice, this approach relies on
``reasonable'' behavior by high-priority processes, which makes reasoning about
systems difficult. Such systems are also brittle and prone to priority
inversions, in which low-priority processes can be inadvertently elevated to
high-priority.

Another approach is to assign tasks of different types appropriate shares of a
cluster. But this approach also fails to capture what we really care about. For
example, we would like high priority processes to expand their use of the
cluster when they need it. 

These problems are considerably exacerbated when we consider preemption,
especially if preemption is costly for the preempted process: when should a low
priority process be allowed to finish, and when should it be preempted?

Microeconomic methods allow for a principled answer to these questions: by
describing ``high priority'' tasks by \emph{how much more} we care about them
finishing faster, we can make principled tradeoffs. 

\paragraph{Data locality} Many tasks run more efficiently when they are
co-located with data. At the moment, this is typically achieved either by
explicit optimization (requiring the scheduler to be aware of detailed
characteristics of the tasks its scheduling) or by heuristics, for example
blocking for 5 seconds before accepting an offered resource which is not local
to data. Again, microeconomic methods provide a principled approach which is
compatible with a lightweight interface between client and scheduler: a client
can estimate how much more slowly a task will run on a machine without the data,
can estimate how likely a machine with the data is to become free, and can
thereby estimate how much of a discount they would need to be offered before
they would accept a machine without their data.

\subsection{Philosophical motivation}
Our primary motivation is strongly to maximize efficiency, rather than other
measures such as utilization, fairness, or strategy-proofness. We view other
characteristics as necessary insofar as they help achieve efficiency. 

In some sense this is problematic, because efficiency is typically unmeasurable.
But there are a number of pragmatically relevant implications:
\begin{enumerate}
  \item Pareto efficiency is a minimal standard for acceptability. To the extent
    that existing approaches fail to be Pareto efficient, this is much more
    damning than a possible failure to be strategy-proof (as long as the
    possibility of manipulation does not in itself destroy our efficiency
    guarantee!)
  \item Even where true prices are unavailable, we think in terms of the real,
    unobserved value of an outcome (such as finishing a computation or meeting a
    deadline). Many existing schemes need to implicitly make judgments about
    these values (because they are making tradeoffs between different outcomes).
    So we don't treat the difficulty of assigning values to outcomes as a
    deal-breaker; to the extent that we already need to make implicit tradeoffs,
    reifying those tradeoffs is only likely to help us improve the situation.
\end{enumerate}

\subsection{Why microeconomic methods?}
The basic paradigm we propose is conceptually simple, though is associated with
many practical difficulties. At each stage where we need to make a decision, we
propose:
\begin{enumerate}
  \item Estimate the value of different outcomes.
  \item Estimate the outcomes that will result from different decisions.
  \item Compute the value of different decisions.
\end{enumerate}
    
This paradigm has the virtue of being modular in a number of important respects:
\begin{enumerate}
  \item The notion of value provides a unifying framework for different
    components to communicate with each other; if I am one component providing a
    service to some client, and I know how valuable different outcomes are to my
    client, I know everything I need to know; my policy does not need to depend
    on anything else about the client.
  \item I can improve and reason about the prediction of outcomes and the
    estimates of value separately. 
  \item It is often possible to compute the value straightforwardly, given the
    values stated  by my clients. For example, if I am scheduling a MapReduce
    instance and know that a 1 second delay costs my client 1 unit, then I can
    estimate the cost of a 1 second delay in one of my tasks simply by
    estimating (or using a crude heuristic for) the probability that this task
    is on the critical path.
\end{enumerate}

\section{Theoretical analysis}

\subsection{Drawbacks of non-economic scheduling}
Most scheduling policies used in practice do not use any economic information: clients request resources, and schedulers allocate those resources. Clients may be assigned priorities and/or budgets which are used by the scheduler to determine how many resources to assign to that client, but requests themselves are an all-or-nothing affair.

These policies can work well when the scheduler is aware of all of the relevant constraints faced by clients. But when constraints are invisible to the scheduler, as is the case in Mesos, this paradigm can become problematic. Approaches like lottery scheduling (though not in fact used in Mesos) can partly get around these problems by ensuring that each client receives a proportional share of every resource type, even if the differences between resources are not known to the scheduler. But these techniques cannot easily ensure that a resource is given to the framework that needs it most.

In this section we prove some theorems that make this difficulty more concrete. Our results illustrate the general theme that the theoretical guarantees of fairness-based schedulers are highly brittle to realistic variations in the model. All of our results are straightforward consequences of well-understood limitations in mechanism design. We feel that these limitations are real and applicable to scheduling problems, and that they show fairly clearly that strong theoretical guarantees like strategy-proofness are not pragmatically desirable. This is in concordance with the dominance of mechanisms which are not strategy-proof in most complex environments. We stress that the results in this section are primarily intended to illustrate the limited applicability of the theoretical analysis of existing schedulers (and hence to excuse the lack of such analysis for our own system) rather than to directly demonstrate practical superiority.

We will use the suggestive term “realistic conditions” to mean a situation in which any of the following occur (i.e., our negative results apply when any of these occur, and the setting is otherwise the simplest setting of static, homogeneous demands for homogeneous resources):

Users have time-varying demands and tasks of varying importance.
Users can apply additional computational resources to speed up a task (e.g. via speculative execution)
Users face deadlines.
Different resources are inhomogeneous, and tasks have inhomogeneous sensitivity to characteristics of resources. Either the scheduler does not make decisions on the basis of these inhomogeneities, or different resources are imperfect substitutes.

We say that a system is “demand-based” if it elicits a demand from each user and makes assignments on the basis of the stated demands. For example, DRF, asset allocation, and CEEI all fall into this class.

Theorem 1. Any demand-based system which is Pareto efficient under realistic conditions must sometimes assign the entire cluster to a single user, an extreme violation of fairness.

Proof: We provide the proof for the time-varying demands case. The other arguments are exactly analogous. We give an extremely simple and artificial case, though analogous behavior would be expected in more realistic cases.

Suppose that there are two users and that at each time each user has enough tasks to consume all of the clusters’ resources. Tasks arrive at multiples of time T and last for T rounds. Moreover, suppose that during each interval of time T each user has an independently and uniformly randomly generated priority which is either “high” or “low.” The only characteristic of these priorities is that, all else equal, they would prefer run one extra high priority task at the expense of one fewer low priority task.

Since the scheduler is demand-based and the user’s priority levels are changing across time, the allocation must be independent of the priority of the users. So if each user receives a non-zero share of the resources, then each user must receive a non-zero share of the resources while they are in the “high” priority state and the other user is in the “low” priority state. 

But now consider the alternative allocation which exchanges these periods, i.e. assigns each user R resources more during their high priority period and R resources fewer during their low priority period. By definition, this alternative allocation is preferred to the real allocation, contradicting Pareto efficiency.

Theorem 2. Under realistic conditions, the implementation of DRF in Mesos is not strategy-proof, Pareto efficient, envy-free, or compatible with the sharing incentive. DRF satisfies a weaker form of strategy-proofness, “look-ahead strategy proofness,” such that a user is incentivized to honestly state their demands given their beliefs about the effect of those demands on the scheduler’s future behavior. 

Proof: For variety, we consider the case of inhomogeneous resources. Again, the proof is identical in any of the given cases. For each property we construct a special situation in which DRF fails to achieve that property.

Strategy-proofness: consider the setting where a framework’s top priority is achieving some particularly valuable resource X which is currently in use, though they have other tasks which they value less. To be most extreme, suppose the user values any outcome in which they receive X higher than any outcome in which they do not. A user can arrange to receive X by strategically passing up resource offers which they could use, in order to maximize their probability of being offered the resource X when it first becomes free. DRF is only “strategy-proof” in the sense that a user who anticipates all future behavior of the scheduler will honestly state whether they want a resource in light of the consequences on the scheduler. 
Pareto-efficiency: This is a corollary of the preceding theorem. In the case of inhomogeneous resources, suppose that two users have streams of tasks, some of which run faster on resources of type A and some of which run faster on resources of type B. Then no demand-based scheduler can ensure that tasks are always placed on the resources on which they run fastest.
Envy-freeness and compatibility with sharing incentive: Suppose that one user consumes ⅓ of the resources in the cluster and another consumes ½ of the resources, but a particular 10% of the resources in the cluster are several times more valuable (to every user) than they appear. This could be based on hardware characteristics to which the scheduler is oblivious. The user who consumes ⅓ of the resources receives all resource offers first, and so claims all 10% of these resources. But this means that this user receives significantly more total value than the other user, violating both envy-freeness and the sharing incentive. 

Theorem 3. The sharing incentive and Pareto efficiency are incompatible for demand-based systems which deal with time-varying demands of varying importance.

Proof: %XXX, Insert proof, it’s super straightforward]
\subsection{Virtues of economic scheduling}
In this section, we show that our system possesses weaker versions of the theoretical guarantees of the theoretical properties that have been ascribed to existing schedulers. In light of the arguments in the preceding section, we hope to illustrate that the two systems are on comparable theoretical footings.

The first theorem considers the setting in which user’s willingness to pay [XXX continue]

Theorem 4. [Sharing incentive] Each user A values the total resources they receive at least as much as they value their share of the cluster (i.e. the amount paid by A divided by the entire cost paid by all users of the cluster).

Proof: Consider a particular user A. During any auction, let $p_A$ be the price that $A$ pays, 
let $p_B$ be the price paid by all other users, let $R_A$ be the resources received by A, and let $R_B$ be the resources received by all other users. The key point is that user A values $R_A$ at at least $p_A$, and values $R_B$ at at most $p_B$. Thus user A values their share of the resources at at least $\frac {p_A}{p_A + p_B}$.

%XXX finish proof, very easy.

Theorem 5. Using a closed currency, if the rate of income of a particular user is p, that user can adopt a bidding strategy which ensures that they receive a p fraction of the cluster during any period of length T, where T is an upper bound for the length of a task (i.e. the maximum of the minimum amount of time after which any task can be revoked).

Proof: 

\section{Market-based Mechanism Design}
\label{sec:design}
\subsection{Setting}
Our mechanism assumes three types of participants: a single \emph{master} for
the cluster which manages a large number of \emph{slaves} serving as a pool of
compute resources. The users of the system are represented by \emph{frameworks},
which communicate with the master to request resources on slaves, and use the
obtained resources to run \emph{tasks} on the slaves.

These participants generally run as daemons on the cluster.  Typically, a single
cluster would consist of up to tens of thousands of machines, each of which
running a slave process. A single master process (or multiple redundant ones
for reliability) communicates directly with all slave processes.  Each slave
process manages the tasks which run on its machine, handling requests to start
and stop various tasks from the master.

The frameworks representing the clusters also run as daemons and register
themselves with the master. They communicate exclusively with the master, which
mediates all needed communication with the slaves, such as details on the task
to be run.

\subsection{Budgets and income}
The master keeps track of a budget for each framework: its current balance and
the amount of income it receives per second. The automatically adjusts a
framework's balance when it performs actions which require payment. Currently,
we allow a framework to incur a negative income (if it is spending beyond its
means) and also a negative balance (if negative income lasts long enough to
remove all of the savings) for simplicity.

\subsection{Resource offers and bidding}
Periodically, when there are resources currently available, the master will
send a \emph{resource offer} message to all registered frameworks to invite them
to participate in an upcoming auction. Each framework then submits a package of
bids, which we can consider a function of the form
\[ \mathcal{B}: \mathbf{T}^n \rightarrow \mathbf{R}. \]
where there are $n$ slaves, $T$ is a task (containing an amount of desired
resources on a single slave) and $\mathbf{R}$ is a willingness to pay for the
list of tasks $\mathbf{T}^n$.

A naive representation of this function would use an exponential amount of
space, given that each of the $n$ values in $\mathbf{T}^n$ can vary
independently. While we can consider several methods for restricting the set of
possible 

\section{Experimental Evaluation}
\label{sec:eval}

To evaluate the performance of the market-based mechanism compared with DRF,
we implemented an event-based simulation of Mesos with both resource scheduling algorithms.
We study the performance of DRF and the market-based mechanism both when pricing information is not available and a default bidding policy must be used, and when pricing information is specified under a specific policy. These simulations were run under a statistical workload generated from a month-long Google cluster trace \cite{TODO}.

\subsection{Simulation Implementation\\}

Our evaluation is conducted through an event-based simulator of Mesos, implemented with both the DRF and the auction-based resource scheduling algorithms. We chose to use simulations because it allowed for simplified evaluation over a variety of environmental variables, such as the number of nodes in the cluster or resources per node, and provided an efficient method for testing various bidding policies. Our simulator was implemented in C++ using approximately 1200 lines of code.

The simulations were run on synthetic workloads generated from the statistical distributions of a month-long Google cluster trace. The Google traces, captured during May 2011, identify cluster jobs as well as the tasks that compose each job. For each task that finishes successfully (without failing or being terminated), we extract the start and finish times, priority information, and normalized resource requests for CPU cores, RAM, and local disk space. The traces indicated that jobs were heavily CPU-bound. We assume tasks run under the amount of resources requested and the task execution times are simply the differences between the finish and start times. We generate workloads that follows the same statistical distribution as this extracted data. The benefit of this statistical workload approach is that we can generate various synthetic workloads that model a real-world workload.

\subsection{Performance Without Pricing\\}

Currently, Mesos jobs and tasks are not attached to economic indicators, since DRF does not rely on any economic-based resource scheduler. Even if a market-based mechanism was used, many cluster jobs and tasks may not be able or willing to provide accurate pricing data. In the absence of this information, our scheme provides a default bidding policy. This ``backwards-compatible" mode must still perform comparable with DRF, especially given the current state of job requests.

To investigate this feature, we ran our simulator using both DRF and the market-based mechanism under the default bidding policy. We ran under two cluster environments, a 25-node cluster to simulate high resource contention, and a 100-node cluster to simulate a light workload. All nodes were homogeneous, with normalized CPU, RAM, and disk resources of 1. We ran using a statistical workload for five computing frameworks, each submitting 20 jobs in parallel. For each environment, we evaluated the resource utilization and job completion times of DRF and the market-based mechanism.

Figures CPU-UTILIZATION and RAM-UTILIZATION show the CPU and RAM utilizations of DRF and the market-based mechanism in the high contention environment. As noted, the Google traces indicated cluster jobs were primarily CPU-bound, and hence the cluster witnesses near-maximal CPU utilization but lower RAM utilization. Figure CPU-UTILIZATION indicates that the utilization achieved by DRF and the default bidding policy are nearly identical. The average DRF utilization is approximately 98\%, while the market-based mechanism is approximately 99\%. Note the graph begins after 10 hours, when consistent maximum contention is reached. Figure RAM-UTILIZATION depicts the memory utilization, showing again the similar performance between DRF and the market-based mechanism. DRF's average memory utilization is TODO\%, while the market-based mechanism's average memory utilization is TODO\%. We do not show the disk utilization because very few jobs in the Google traces used local disk space (presumably instead relying on the Google File System).

Figure CPU-UTILIZATION-LOW shows the CPU utilizations of DRF and the market-based mechanism in a low contention environment, when no resource is fully utilized. As we can see, DRF and the market-based mechanism perform nearly identical. Both policies have an average CPU utilization of 44\%. We elide the memory and disk utilization figures because they repeat the same story, but are less interesting due to the lower RAM and disk resource requirements of jobs.

Finally, Figure JOB-COMPLETION displays the progression of workload completion under both scheduling policies. When all tasks of a job are completed, the job is considered finished. The graph shows the fraction of complete jobs as time progresses. Again, both policies result in similar behaviour, completing the workload in nearly identical fashion.

All of these results are gathered when the market-based mechanism uses a default bidding policy because the job requests provide no economic information. This shows that the market-based mechanism is ``backwards-compatible" by providing comparable performance to DRF when pricing information is not used. This result is not completely surprising though, because the default bidding policy behaves similarly to DRF. DRF awards resources to the framework with the least dominant share. Under our default bidding policy, frameworks bid a proportion of their current budget. The framework that has been awarded the least resources will have the highest budget (through less spending on resources), and will be able to bid the highest. Hence the low-resource framework will be awarded the next resource offer, similarly to DRF.

\subsection{Performance With Pricing\\}

When pricing information is provided, we show that the market-based mechanism can provide better performance than in DRF. As discussed in Section TODO, there are many situations in which prices can be used to better encapsulate a framework's preferences. As a case study, we investigate one particular scenario of inhomogeneous performing resources. 

In this case, we assume certain nodes have a special resource that significantly improves job performances, while the remaining nodes do not. In a realistic case, this special resource could be a GPU, which can significantly reduce CPU-bounded job completion times. While most jobs will see some performance benefits from running on a GPU, some jobs may be specifically crafted for optimal performance on a GPU, and hence will see the largest performance improvement if run on a GPU.

The current Mesos implementation does not recognize this special sort of inhomogeneous performing resource. However, if a framework is aware of the inhomogeneous performing resource, it can schedule tasks appropriately on slaves when offered, such that the tasks that benefit most from special resources will be scheduled on special resource slaves. For the simulations, we have implemented the framework scheduler in such a manner. For the market-based mechanism, we use a bidding policy proportional to the default bidding policy. However, when a task can benefit from a special slave by a speedup factor $s$, the default bid is multiplied by $\frac{1}{s}$, resulting in a bid that is inversely proportional to the speedup factor.

We simulate this scenario using a 64-node cluster, where 16 nodes have the special resource. Jobs from the statistical workload are uniformly randomly assigned one of two speedup factors, such that all tasks in a job experience the same speedup factor. The ``slow" speedup reduces task runtime by 10\%, while the ``fast" speedup reduces task runtime by 90\%. All tasks with a ``fast" speedup are marked as preferring a special resource slave, and a framework in DRF Mesos will schedule a ``fast" task on a special resource slave if possible. The workload was again generated for five frameworks, with up to 20 jobs running in parallel.

Figure JOB-COMPLETION depicts the time progression of job completions, where a job is considered finished when all of its tasks are finished. As the figure depicts, the market-based mechanism is consistently ahead of DRF at all times, and finishes all jobs in about 20\% less time than DRF. Looking in detail, we find that DRF schedules only 30\% of ``fast" tasks on special resource slaves, while the market-based mechanism schedules 60\%. One might note that around the eighth hour, it appears DRF catches up to the market-based mechanism before falling behind again. This effect is caused by a long running job that the market-based mechanism schedules earlier than DRF. Its completion progresses the fraction of job completion minimally yet accounts for a long period of time. This allows DRF, which is still completing earlier short-lived tasks, to momentarily catch up in the fraction of job completion. When DRF schedules the same long-lived job at a latter point, it once again lags behind the market-based mechanism.

Hence, in the case of inhomogeneous performing resources, the market-based mechanism can allow frameworks to better express their preferences than DRF, resulting in better performance for the market-based mechanism. DRF only allows a framework to specifically target special resource slaves when the framework is offered, whereas the market-based mechanism allows frameworks who benefit most from a special resource slave to outbid others. By better scheduling jobs on machines that best complete them, we see the overall performance improvements of the market-based mechanism.

\section{Related Work}
\subsection{Grid computing}
Job schedulers like TORQUE \cite{torque}, Condor \cite{condor}, and LSF
\cite{lsf} have long been used for solving the problem of resource allocation
within a high-performance-computing (HPC) or grid-computing environment.
For the most part, they do not use a two-level scheduling model like we do but
instead operate as a queue for predefined jobs which are run as resources become
available. Machines and job requirements also tend to show more homogeniety
within such environments (for example, the ratio between CPU, memory, and disk),
so sthe amount of optimization possible with pricing is lower.

\subsection{Cloud and datacenter computing}
Our work depends heavily on Mesos~\cite{mesos}, as we use its model of
resources, tasks, and frameworks as the basis for our own. We also use Mesos as
a baseline for purposes of comparison, as we have already extensively discussed.

Jockey~\cite{jockey} attempts to solve the specific problem of adjusting
resource allocation in the datacenter environment to ensure important jobs
finish within deadlines. They do not directly modify the scheduler for this but
instead adjust the priority of each job as needed within an existing system.
With our work, we attempt to create a general-purpose system which can solve not
only the problem of deadlines, but also allow expression of many different kinds
of demands and have them all interact nicely with each other without needing
specialized logic to keep track of each one explicitly.

\subsection{Resource allocation}
Dominant resource fairness~\cite{drf} 

\subsection{Economics/pricing in computer systems}
Lottery scheduling

\subsection{Economics/pricing for resource allocation}
Many previous researchers have proposed systems which use ideas from economics
to assign resources in computer systems.  Analogous to our datacenter
implementation, \cite{stoica94} uses prices, budgets, and auctions for
scheduling within a single machine. It ensures some level of fairness and
allows low-income but mostly-idle users to occasionally spend more money beyond
their income for urgent jobs by drawing upon their account balance. However,
resource demands are much more homogeneous in this environment than within a
datacenter, so the usefulness of prices as a vehicle for expressing prices are
more limited. Spawn \cite{spawn} uses auctions to rent time-slices of idle
workstations; rather than using prices as a priority-indicating mechanism
automatically set based on application requirements, this work places more
emphasis on using it for reimbursement where users manually allocate budgets to
jobs. Tycoon \cite{tycoon} also uses auctions to divide up the cluster's
resources among competing users, except the auctions are used for CPU scheduling
within each individual machine rather than for setting bounds on allowed
resource usage of each job, and has no provision for the presence of
multiple contentious resources.

More generally, fair division is the problem in economics of dividing a set of
resources equitably among a set of people. Our problem differs somewhat from
fair division in that we don't seek to allocate every bit of the cluster to the
set of users. We assume that a user might be happy to have received nothing if
their value of gaining resources was too low compared to the others, since the
user keeps some amont of money unspent as a a result.

One of the economic fair division solutions is CEEI~\cite{moulin2004fair}, or
competitive equilibrium from equal incomes. This mechanism emulates a market
where participants initially receive an equal amount of resources and trade some
of them until each has the desired amount of resources. CEEI's use of a market
differs significantly from ours in that we do not use it only as a vehicle for
achieving a fair allocation, but also for frameworks to express their
preferences in a fine-grianed way using prices and willingness to pay.

\section{Discussion and Future Work}
\label{sec:future}


\section{Conclusion}
\label{sec:conclusion}

\section{Acknowledgement}

We thank Professors Anthony Joseph and John Kubiatowicz for guidance and numerous suggestions throughout the course of this project. Also, we thank Charles Reiss for providing helpful tips on how best to use the Google cluster traces. Finally, we thank Ali Ghodsi for providing meaningful discussion into the theory behind our market-based mechanism.

%
% The following two commands are all you need in the
% initial runs of your .tex file to
% produce the bibliography for the citations in your paper.
\bibliographystyle{abbrv}
\bibliography{paper}  % sigproc.bib is the name of the Bibliography in this case
% You must have a proper ".bib" file
%  and remember to run:
% latex bibtex latex latex
% to resolve all references
%
% ACM needs 'a single self-contained file'!
%
%APPENDICES are optional
%\balancecolumns
%\appendix
%Appendix A
%\section{Headings in Appendices}
%The rules about hierarchical headings discussed above for
%the body of the article are different in the appendices.
%In the \textbf{appendix} environment, the command
%\textbf{section} is used to
%indicate the start of each Appendix, with alphabetic order
%designation (i.e. the first is A, the second B, etc.) and
%a title (if you include one).  So, if you need
%hierarchical structure
%\textit{within} an Appendix, start with \textbf{subsection} as the
%highest level. Here is an outline of the body of this
%document in Appendix-appropriate form:
%\subsection{Introduction}
%\subsection{The Body of the Paper}
%\subsubsection{Type Changes and  Special Characters}
%\subsubsection{Math Equations}
%\paragraph{Inline (In-text) Equations}
%\paragraph{Display Equations}
%\subsubsection{Citations}
%\subsubsection{Tables}
%\subsubsection{Figures}
%\subsubsection{Theorem-like Constructs}
%\subsubsection*{A Caveat for the \TeX\ Expert}
%\subsection{Conclusions}
%\subsection{Acknowledgments}
%\subsection{Additional Authors}
%This section is inserted by \LaTeX; you do not insert it.
%You just add the names and information in the
%\texttt{{\char'134}additionalauthors} command at the start
%of the document.
%\subsection{References}
%Generated by bibtex from your ~.bib file.  Run latex,
%then bibtex, then latex twice (to resolve references)
%to create the ~.bbl file.  Insert that ~.bbl file into
%the .tex source file and comment out
%the command \texttt{{\char'134}thebibliography}.
%% This next section command marks the start of
%% Appendix B, and does not continue the present hierarchy
%\section{More Help for the Hardy}
%The acm\_proc\_article-sp document class file itself is chock-full of succinct
%and helpful comments.  If you consider yourself a moderately
%experienced to expert user of \LaTeX, you may find reading
%it useful but please remember not to change it.
\balancecolumns
% That's all folks!
\end{document}
